\documentclass[10pt, a4paper]{article}
\usepackage[paper=a4paper, left=1.5cm, right=1.5cm, bottom=1.5cm, top=3.5cm]{geometry}
\usepackage[latin1]{inputenc}
\usepackage[T1]{fontenc}
\usepackage[spanish]{babel}
\usepackage{indentfirst}
\usepackage{fancyhdr}
\usepackage{latexsym}
\usepackage{lastpage}
\usepackage{aed2-symb,aed2-itef,aed2-tad, caratula}
\usepackage[colorlinks=true, linkcolor=blue]{hyperref}
\usepackage{calc}

\newcommand{\f}[1]{\text{#1}}
\renewcommand{\paratodo}[2]{$\forall~#2$: #1}

\sloppy

%  \hypersetup{%
%   % Para que el PDF se abra a p�gina completa.
%   pdfstartview= {FitH \hypercalcbp{\paperheight-\topmargin-1in-\headheight}},
%   pdfauthor={C�tedra de Algoritmos y Estructuras de Datos II - DC - UBA},
%   pdfkeywords={TADs b�sicos},
%   pdfsubject={Tipos abstractos de datos b�sicos}
%  }

\parskip=5pt % 10pt es el tama�o de fuente

% Pongo en 0 la distancia extra entre �temes.
\let\olditemize\itemize
\def\itemize{\olditemize\itemsep=0pt}

% Acomodo fancyhdr.
\pagestyle{fancy}
\thispagestyle{fancy}
\addtolength{\headheight}{1pt}
\lhead{Algoritmos y Estructuras de Datos II}
\rhead{$1^{\mathrm{er}}$ cuatrimestre de 2015}
\cfoot{\thepage /\pageref{LastPage}}
\renewcommand{\footrulewidth}{0.4pt}

\author{Algoritmos y Estructuras de Datos II, DC, UBA.}
\date{}
\title{Tipos abstractos de datos b�sicos}

\begin{document}


% Estos comandos deben ir antes del \maketitle
\materia{Algoritmos y Estructuras de Datos II} % obligatorio
\submateria{Primer Cuatrimestre de 2015} % opcional
\titulo{Trabajo Pr�ctico 1} % obligatorio
\subtitulo{Especificaci�n} % opcional


\integrante{BENITEZ, Nelson}{945/13}{nelson.benitez92@gmail.com} % obligatorio 
\integrante{ROIZMAN, Violeta}{273/11}{violeroizman@gmail.com} % obligatorio 
\integrante{V�ZQUEZ, J�sica}{318/13}{jesis\_93@hotmail.com} % obligatorio 
\integrante{ZAVALLA, Agust�n}{670/13}{nkm747@gmail.com} % obligatorio 
%Pagina de titulo e indice
\thispagestyle{empty}

\maketitle
\tableofcontents

\newpage


\section{TAD \tadNombre{AS}}
\begin{tad}{\tadNombre{AS}}
\tadGeneros{as}

\tadIgualdadObservacional{dc}{dc'}{dcnet}{}

\tadUsa{\tadNombre{campus}}

\tadExporta{\tadNombre{}}

\tadObservadores
\tadOperacion{campus}{as}{campus}{}
\tadOperacion{seguridad}{as}{conj(seguridad)}{}
\tadOperacion{hayEst?}{as/a,pos/p}{bool}{$posValida(campus(a),p)$}
\tadOperacion{hayHippie?}{as/a,pos/p}{bool}{$posValida(campus(a),p)$}
\tadOperacion{posSeg}{as/a,seg/s}{pos}{$s \in seguridad(a)$}
\tadOperacion{\#capturas}{as/a,seg/s}{nat}{$s \in seguridad(a)$}
\tadOperacion{\#sanciones}{as/a,seg/s}{nat}{$s \in seguridad(a)$}

\tadGeneradores

\tadOperacion{nueva}{campus,conj(seguridad)}{as}{(\paratodo{e}{segs}) posValida(c,pos(e)) $\land$ (\paratodo{s,s1}{segs}) id(s)!=id(s1) $\implies$ pos(s)!=pos(s1)}
\tadOperacion{moverEst}{as/a,pos/pe,pos/pd}{as}{}
\tadOperacion{nuevoEst}{as/a,pos/p}{as}{$posValida(campus(a),pe) \yluego hayEst?(a,p) \land adyacente(campus(a),pe,pd) \land posValidaPersona(as,pd)$}
\tadOperacion{nuevoHippie}{as/a,pos/p}{as}{$posIngreso(campus(a),p) \land posValidaPersona(a,p)$}
\tadOperacion{sacarEst}{as/a,pos/p}{as}{$posValida(campus(a),p) \yluego hayEst?(a,p) \land posIngreso(a,p)$}

\tadOtrasOperaciones

\tadOperacion{haySeg?}{as/a,pos/p}{bool}{}
\tadOperacion{adyacente}{as/a,pos/pe,pos/pd}{bool}{}
\tadOperacion{posValidaPersona}{as/a,pos/p}{bool}{}
\tadOperacion{posIngreso}{as/a,pos/p}{bool}{}

\tadAxiomas[\paratodo{dcnet}{dc}, \paratodo{red}{r}, \paratodo{paqueteID}{p_{1},p_{2}}, \paratodo{compuID}{c_{1},c_{2},c_{3}},\\\paratodo{secu(tupla(compuID,interfaz)}{camino},\\\paratodo{conj(paqueteID)}{cpaq}]

%que es esto? Es para que quede alineado y bonito y no se vaya de la hoja.
\tadAlinearAxiomas{\#paquetesEnviados(ingresarPaquete($dc$,$p_1$,$c_1$,$c_2$),$c_3$),d 923)}
\tadAlinearFunciones{encolarPaquetes(conjPaquetes,colaPrior)}{}

\tadAxioma{red(nueva($r$))}{$r$}

\end{tad}


\section{TAD \tadNombre{campus}}
\begin{tad}{\tadNombre{campus}}
\tadGeneros{campus}

\tadUsa{\tadNombre{campus}}

\tadExporta{\tadNombre{}}

\tadObservadores
\tadOperacion{alto}{campus}{nat}{}
\tadOperacion{ancho}{campus}{nat}{}
\tadOperacion{obstaculos}{campus}{conj(pos)}{}

\tadGeneradores

\tadOperacion{nuevo}{nat/ancho,nat/alto,conj(pos)/obst}{campus}{$1 \le ancho \land 1 \le alto$ $\land$ (\paratodo{pos}{p}) $p \in obst \impluego posValida(c,p)$}

\tadOtrasOperaciones

\tadOperacion{adyacente}{as/a,pos/pe,pos/pd}{bool}{$posValida(c,pe) \land posValida(c,pd)$}
\tadOperacion{posValida}{as/a,pos/p}{bool}{}
\tadOperacion{posIngreso}{as/a,pos/p}{bool}{}

\tadAxiomas[\paratodo{nat}{alto}, \paratodo{nat}{ancho}, \paratodo{conj (pos)}{obst}
\\\paratodo{pos}{p _1} \paratodo {pos}{p _2}]

%que es esto? Es para que quede alineado y bonito y no se vaya de la hoja.
\tadAlinearAxiomas{\#paquetesEnviados(ingresarPaquete($dc$,$p_1$,$c_1$,$c_2$),$c_3$),d 923)}
\tadAlinearFunciones{encolarPaquetes(conjPaquetes,colaPrior)}{}

\tadAxioma{alto(nuevo($ancho$,$alto$,$obst$))}{$alto$}
\tadAxioma{ancho(nuevo($ancho$,$alto$,$obst$))}{$ancho$}
\tadAxioma{obstaculos(nuevo($ancho$,$alto$,$obst$))}{$obst$}
\tadAxioma{posValida(nuevo($ancho$,$alto$,$obst$),$p_1$)}{$\pi_1(p_1)<ancho \land \pi_2(p_1)<alto$}
\tadAxioma{adyacente(nuevo($ancho$,$alto$,$obst$),$p _1$,$p _2$)}{$(\pi_1(p_1)=\pi_1(p_2)-1 \lor\pi_1(p_1)=\pi_1(p_2)+1)\land (\pi_2(p_1)=\pi_2(p_2)-1 \lor\pi_2(p_1)=\pi_2(p_2)+1)$}
\tadAxioma{posValida(nuevo($ancho$,$alto$,$obst$),$p_1$)}{$\pi_2(p_1)=alto-1 \lor \pi_2(p_1)=0$}
\end{tad}


\end{document}
