\documentclass[10pt, a4paper]{article}
\usepackage[paper=a4paper, left=1.5cm, right=1.5cm, bottom=1.5cm, top=3.5cm]{geometry}
\usepackage[latin1]{inputenc}
\usepackage[T1]{fontenc}
\usepackage[spanish]{babel}
\usepackage{indentfirst}
\usepackage{fancyhdr}
\usepackage{latexsym}
\usepackage{lastpage}
\usepackage{aed2-symb,aed2-itef,aed2-tad, caratula}
\usepackage[colorlinks=true, linkcolor=blue]{hyperref}
\usepackage{calc}

\newcommand{\f}[1]{\text{#1}}
\renewcommand{\paratodo}[2]{$\forall~#2$: #1}

\sloppy

%  \hypersetup{%
%   % Para que el PDF se abra a p�gina completa.
%   pdfstartview= {FitH \hypercalcbp{\paperheight-\topmargin-1in-\headheight}},
%   pdfauthor={C�tedra de Algoritmos y Estructuras de Datos II - DC - UBA},
%   pdfkeywords={TADs b�sicos},
%   pdfsubject={Tipos abstractos de datos b�sicos}
%  }

\parskip=5pt % 10pt es el tama�o de fuente

% Pongo en 0 la distancia extra entre �temes.
\let\olditemize\itemize
\def\itemize{\olditemize\itemsep=0pt}

% Acomodo fancyhdr.
\pagestyle{fancy}
\thispagestyle{fancy}
\addtolength{\headheight}{1pt}
\lhead{Algoritmos y Estructuras de Datos II}
\rhead{$1^{\mathrm{er}}$ cuatrimestre de 2015}
\cfoot{\thepage /\pageref{LastPage}}
\renewcommand{\footrulewidth}{0.4pt}

\author{Algoritmos y Estructuras de Datos II, DC, UBA.}
\date{}
\title{Tipos abstractos de datos b�sicos}

\begin{document}


% Estos comandos deben ir antes del \maketitle
\materia{Algoritmos y Estructuras de Datos II} % obligatorio
\submateria{Primer Cuatrimestre de 2015} % opcional
\titulo{Trabajo Pr�ctico 1} % obligatorio
\subtitulo{Especificaci�n} % opcional


\integrante{BENITEZ, Nelson}{945/13}{nelson.benitez92@gmail.com} % obligatorio 
\integrante{ROIZMAN, Violeta}{273/11}{violeroizman@gmail.com} % obligatorio 
\integrante{V�ZQUEZ, J�sica}{318/13}{jesis\_93@hotmail.com} % obligatorio 
\integrante{ZAVALLA, Agust�n}{670/13}{nkm747@gmail.com} % obligatorio 
%Pagina de titulo e indice
\thispagestyle{empty}

\maketitle
\tableofcontents

\newpage


\section{TAD \tadNombre{DCNet}}
\begin{tad}{\tadNombre{DCNet}}
\tadGeneros{dcnet}

\tadIgualdadObservacional{dc}{dc'}{dcnet}{red(dc) $\igobs$ red(dc') $\land$ 
	\\\\(\paratodo{compuID}{c}{,c$\in$compus(red(dc))$\land$c$\in$compus(red(dc'))})
	\\(colaPaquetes(dc,c) $\igobs$ colaPaquetes(dc',c)) $\land$
	\\\\(\paratodo{paqueteID}{p}{,p$\in$paquetes(dc)$\land$p$\in$paquetes(dc')}) 
	\\(origenPaquete(dc,p)$\igobs$origenPaquete(dc',p)) $\land$
	\\\\(\paratodo{paqueteID}{p}{,p$\in$paquetes(dc)$\land$p$\in$paquetes(dc')}) 
	\\(destinoPaquete(dc,p)$\igobs$destinoPaquete(dc',p)) $\land$
	\\\\(\paratodo{compuID}{c}{,c$\in$compus(red(dc))$\land$c$\in$compus(red(dc'))})
	\\(\#paquetesEnviados(dc,c)$\igobs$\#paquetesEnviados(dc',c)) $\land$
	\\\\(\paratodo{paqueteID}{p}{,p$\in$paquetes(dc)$\land$p$\in$paquetes(dc')}) 
	\\(prioridad(dc,p)$\igobs$prioridad(dc',p))}

\tadUsa{\tadNombre{Nat, Bool, Secu, Red, PaqueteID, compuID, Interfaz}}

\tadExporta{\tadNombre{observadores b�sicos, generadores, caminoRecorrido, \#paquetesEnEspera, \\laQueM�sEnvi�}}

\tadObservadores
\tadOperacion{red}{dcnet}{red}{}

\tadGeneradores

\tadOperacion{nueva}{red}{dcnet}{}
\tadOperacion{ingresarPaquete}{dcnet/dc,paqueteID/p,prioridad/pr,compuID/c_1,compuID/c_2}{dcnet}
{$\neg(c_1=c_2) \land \neg(p \in paquetes(dc)) \land c_1 \in compus(red(dc)) \land c_2 \in  compus(red(dc)) \yluego existeCamino?(red(dc), c_1, c_2)$}
\tadOtrasOperaciones

\tadOperacion{caminoRecorrido}{dcnet/dc,paqueteID/p}{secu(tupla(compuID, interfaz))}{p $\in$ paquetes(dc)}

\tadAxiomas[\paratodo{dcnet}{dc}, \paratodo{red}{r}, \paratodo{paqueteID}{p_{1},p_{2}}, \paratodo{compuID}{c_{1},c_{2},c_{3}},\\\paratodo{secu(tupla(compuID,interfaz)}{camino},\\\paratodo{conj(paqueteID)}{cpaq}]

%que es esto? Es para que quede alineado y bonito y no se vaya de la hoja.
\tadAlinearAxiomas{\#paquetesEnviados(ingresarPaquete($dc$,$p_1$,$c_1$,$c_2$),$c_3$),d 923)}
\tadAlinearFunciones{encolarPaquetes(conjPaquetes,colaPrior)}{}

\tadAxioma{red(nueva($r$))}{$r$}

\end{tad}

\end{document}
